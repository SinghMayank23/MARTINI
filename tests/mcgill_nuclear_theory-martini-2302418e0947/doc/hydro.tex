\chapter{Hydrodynamic background}
\label{hydro}\index{Hydrodynamics}\index{background}
\section{Hydro data}
Hydro data is given for $tau$, $x$, $y$, and $z$ (for the 3D case). 
The original Lagrangian 3D hydro data was converted from $\eta$ to $z$.
The flow velocities are still $v_x$, $v_y$, $v_\eta$ though and will 
be converted in the main program.
\subsection{General parameters}
\[
\framebox[16cm]{
\parbox{15cm}{Parameters in the file #./main/xmldoc/Hydro.xml#:
\begin{itemize}
\item
#Hydro:WhichHydro#: choose a hydro evolution model:
\begin{enumerate}
   \item Kolb 2+1D Hydro (see page \pageref{kolbhydro} for parameters and page \pageref{kolbdata} for data structure)
   \item Eskola 2+1D Hydro (see page \pageref{eskolahydro} for parameters and page \pageref{eskoladata} for data structure)
   \item Nonaka 3+1D Hydro (see page \pageref{nonakahydro} for parameters and page \pageref{nonakadata} for data structure)
   \item Schenke 3+1D Hydro (see page \pageref{schenkehydro} for parameters and page \pageref{schenkedata} for data structure)
\end{enumerate}
\item #Hydro:Subset#: choose a parameter subset - only available for #WhichHydro#=4.
\item #Hydro:tau0#: initial $\tau_0$ [fm] in the data file.
\item #Hydro:taumax#: final time $\tau_{\rm max}$ [fm] in the files, can be changed to read in less data.
\item #Hydro:dtau#: step size in time $\Delta\tau$ [fm], as used in the data files
\item #Hydro:xmax#: maximal $x$ and $y$ [fm] (same for both transverse directions). Data runs from $-x_{max}$ to $+x_{max}-\Delta x$ such that there are $(2 x_{max})/\Delta x$ steps, including $x=0$. 
\item #Hydro:zmax#: maximal $z$ [fm] - only relevant in the 3D hydro case. Principally the same as #xmax# but the value can of course be different.
\item #Hydro:dx#: step size in both transverse directions $\Delta x=\Delta y$ [fm], as used in the data files.
\item #Hydro:dz#: step size $\Delta z$ [fm] in the z direction as used in the data files - only relevant in the 3D hydro case.
\item #Hydro:Tfinal#: temperature [GeV] at which to stop the evolution of the jet. Usually $T_{\rm crit}$.
\end{itemize}
}}\]
~\\
\subsection{Parameter values for the different hydro models}
The following parameter values are currently used in the three different cases implemented. These are due to change - also more models can be implemented. All parameter names are preceded by #Hydro:# .
\subsubsection{Kolb 2+1D Hydro}
\label{kolbhydro}
\begin{tabular}{l l l}
   #whichHydro# & 1     &    ~  \\
   #tau0#       & 0.6   & # fm# \\
   #taumax#     & 17.16 & # fm# \\
   #dtau#       & 0.04  & # fm# \\
   #xmax#       & 10.   & # fm# \\
   #dx#         & 0.1   & # fm# \\
   #Tfinal#     & 0.16  & # GeV# 
\end{tabular}

\subsubsection{Eskola 2+1D Hydro}
\label{eskolahydro}
\begin{tabular}{l l l}
   #whichHydro# & 2         &    ~  \\
   #tau0#       & 0.17      & # fm# \\
   #taumax#     & 30.171565 & # fm# \\
   #dtau#       & 0.1       & # fm# \\
   #xmax#       & 15.       & # fm# \\
   #dx#         & 0.1       & # fm# \\
   #Tfinal#     & 0.16      & # GeV# 
\end{tabular}

\subsubsection{Nonaka 3+1D Hydro}
\label{nonakahydro}
\begin{tabular}{l l l}
   #whichHydro# & 3         &    ~  \\
   #tau0#       & 0.6       & # fm# \\
   #taumax#     & 17$^*$    & # fm# \\
   #dtau#       & 0.1       & # fm# \\
   #xmax#       & 10.       & # fm# \\
   #zmax#       & 20.       & # fm# \\
   #dx#         & 0.25      & # fm# \\
   #dz#         & 0.5       & # fm# \\
   #Tfinal#     & 0.16      & # GeV# 
\end{tabular}

\subsubsection{Schenke/Jeon/Gale 3+1D Hydro $^{**}$}
\label{schenkehydro}
\begin{tabular}{l l l}
   #whichHydro# & 4         &    ~  \\
   #Subset#     & any int   &    ~  \\
   #tau0#       & 0.55      & # fm# \\
   #taumax#     & 20        & # fm# \\
   #dtau#       & 0.1       & # fm# \\
   #xmax#       & 10.       & # fm# \\
   #zmax#       & 20.       & # fm# \\
   #dx#         & 0.5       & # fm# \\
   #dz#         & 0.5       & # fm# \\
   #Tfinal#     & 0.12      & # GeV# 
\end{tabular}\\~\\

$^*$ this is the absolute maximum. You can choose smaller values. 
That would save time and memory. Maximal values may change for different impact
parameters. Usually is no problem because for larger impact parameters the evolution
will stop earlier such that MARTINI will not try to access data that is not there...

The Nonaka 3+1D hydro is available for four impact parameters at the moment:
$b=2.4, 4.5, 6.3$ and $7.5\,{\rm fm}/c$. Every data set is stored in 
a separate folder under #./hydro/nonaka/bx.x#.
Also, note that for the three larger impact parameters available 
($b=4.5, 6.3$ and $7.5\,{\rm fm}/c)$, the data files available 
only contain information up to $\tau=20\,{\rm fm}/c=$#taumax#.

$^{**}$ these are just typical values - please check the #input# file in the correct directory and for
the correct subset and adjust accordingly. 
The Schenke 3+1D hydro allows for including different subsets of parameters for a given impact 
parameter. The parameters used for the subset using #evolutionN.dat#, where #N# is the number
of the subset, are stored in the same directory in the file #inputN#.



\newpage
\subsection{Data structure}
\label{structure}\index{Hydro data structure}
Let us briefly discuss the ordering in the data files and how to read them in.
\subsubsection{Kolb 2+1D Hydro}
\label{kolbdata}\index{Hydrodynamics!2+1D Hydro!Kolb}
The data is stored in the subfolder #./hydro#.
Each data file corresponds to one time step. There is one for the temperature 
#Txxx.dat#, one for the flow velocity in the $x$ direction at $\eta=0$ 
#VXxxx.dat#, and one for the flow velocity in the $y$ direction at $\eta=0$.
Each file contains information for all $x$ and $y$ values, where $x$ changes in 
the inner loop, $y$ in the outer. When read in , the data will be ordered like this:\\

\begin{boxedverbatim}
 position = ix+ixmax*(iy+iymax*itau) 
\end{boxedverbatim}

~\\
where each step is of size #dx# and we start at #-xmax#, for both $x$ and $y$.
~\\

Here is an example code snippet for reading one file:

\begin{boxedverbatim}
    // temperature
    fin.open(file[0].c_str(),ios::in);
    int ik = 0;
    while ( !fin.eof() )
      {
        ik++;
        fin >> T;
        newCell.T(T);
        if (ik<=ixmax*ixmax) lattice->push_back(newCell);
      }
    fin.close();
    // flow in x direction
    fin.open(file[1].c_str(),ios::in);
    position=ixmax*ixmax*i;
    ik = 0;
    while ( !fin.eof() )
      {
        ik++;
        fin >> vx;
        if (ik<=ixmax*ixmax) 
          {
            lattice->at(position).vx(vx);
            position++;
          }
      }
   fin.close();
\end{boxedverbatim}

\newpage
\subsubsection{Eskola 2+1D Hydro}
\label{eskoladata}\index{Hydrodynamics!2+1D Hydro!Eskola}
The data is stored in the sub-folder #./hydro/RHIC200#. There is one file for each the 
temperature, and the flow velocity in the radial direction (this is for imoact parameter $b=0$ only, so the data is isotropic in the transverse plane), #TMATIn# and #VrMATin#.
The files contain the whole evolution in the coordinates $r$ and $\tau$. $r$ changes in the inner loop, $\tau$ in the outer loop.
Also here, the data will be structured like this after read in:\\

\begin{boxedverbatim}
 position = ix+ixmax*(iy+iymax*itau) 
\end{boxedverbatim}

~\\

Here is an example code to read data from the files:

\begin{boxedverbatim}
   // temperature
   fin.open(file[0].c_str(),ios::in);
   int i = 0;
   int ir, itau;
   while ( !fin.eof() )
     {
       fin >> T;
       ir=(i)%ixmax;
       itau=floor((i)/ixmax);
       if (itau<itaumax && ir<ixmax) TofRandTau[itau][ir] = T;
       i++;
     }
   fin.close();
   for(itau=0; itau<itaumax; itau++)
   for(int ix=0; ix<ixmax; ix++)
   for(int iy=0; iy<ixmax; iy++)
      {
        x = -xmax + ix*dx;
        y = -xmax + iy*dx;
        r=sqrt(x*x+y*y);
        ir=floor(r/dx);
        if ( ir<ixmax ) T=TofRandTau[itau][ir];
        else T=0.;
        newCell.T(T);
        lattice->push_back(newCell);
      }
\end{boxedverbatim}

~\\
Note that we have to convert from $r$ to $x$ and $y$ to get the same structure as for the other hydro data. For more code see the file #./main/src/HydroSetup.cpp#.

\newpage
\subsubsection{Nonaka 3+1D Hydro}
\label{nonakadata}\index{Hydrodynamics!3+1D Hydro!Nonaka}
The structure is similar to that of the Kolb data.
The data is stored in the subfolder #./hydro/nonaka/bx.x#, where #bx.x# is the sub foldercontaining the data for a particular impact parameter ($b=$#x.x# in fm, e.g. $2.4\,\mathrm{fm}$).
Each data file corresponds to one time step. There is one for the temperature 
#Tx.dat#, and one for the flow velocities in the $x$, $y$ and $\eta$ direction #Vx.dat# 
(#x# is the number of the time step). Note: It is really $\tilde{v}_\eta$ that is stored, despite the coordinates are $x$, $y$,
and $z$.
The temperature file also includes a column indicating the fraction of QGP present in 
this cell - this is relevant in the mixed phase.
Each file contains information for all $x$, $y$, and $z$ values, where $x$ changes 
in the inner loop, then $y$, then $z$ in the outer. When read in, the data will be 
ordered like this:\\

\begin{boxedverbatim}
 position = ix + ixmax*(iy+ixmax*(iz+izmax*itau))
\end{boxedverbatim}

~\\
So there is just one more dimension. 
~\\

Let's look at the code that reads the data:

\begin{boxedverbatim}
   // read in temperature
   fin.open(file[0].c_str(),ios::in);
   int ik = 0;
   while ( !fin.eof() )
     {
       ik++;
       fin >> T;
       fin >> QGPfrac;
       newCell.T(T);
       newCell.QGPfrac(QGPfrac);
       if (ik<=ixmax*ixmax*izmax) lattice->push_back(newCell);
     }
   fin.close();  
   // read in flow in x, y, eta direction
   fin.open(file[1].c_str(),ios::in);
   ik = 0;
   position=ixmax*ixmax*izmax*i;
   while ( !fin.eof() )
     {
       ik++;
       fin >> vx;
       fin >> vy;
       fin >> veta;
       if (ik<=ixmax*ixmax*izmax)
         {
           lattice->at(position).vx(vx);
           lattice->at(position).vy(vy);
           lattice->at(position).vz(veta);
           position++;
         }
     }
   fin.close();
\end{boxedverbatim}

\subsubsection{Schenke/Jeon/Gale 3+1D Hydro}
\label{schenkedata}\index{Hydrodynamics!3+1D Hydro!Schenke}
The structure is similar to that of the Nonaka data.
The data is stored in the subfolder #./hydro/schenke/bx.x#, where #bx.x# is the sub folder 
containing the data for a particular impact parameter ($b=$#x.x# in fm, e.g. $2.4\,\mathrm{fm}$).
All data is stored in the file #evolution.dat# or #evolutionN.dat#, where #N# indicates the parameter subset
(#N#$>1$). The information on what the used parameters are is stored in the files #input# or #inputN#.
The columns in the files #evolution.dat# and #evolutionN.dat# are in this order:\\
temperature, QGP fraction, $v_x$, $v_y$, $v_z$.



\newpage
\section{Using the hydro background in the evolution}
\index{background}
The data is read into the #lattice#, a #vector# of #HydroCell# objects, which are defined in #HydroCell.h#:\\
~\\

\begin{boxedverbatim}
class HydroCell
{
 private:
  double itsT;
  Vec4   itsV;
  double itsQGPfrac;

 public:
  HydroCell(){};//constructor
  ~HydroCell(){};//destructor

  // flow velocity
  Vec4 v() const { return itsV; };
  void v(Vec4 value) { itsV=value; };

  double vx() const { return itsV.px(); };
  double vy() const { return itsV.py(); };
  double vz() const { return itsV.pz(); };
  void vx( double value ) { itsV.px(value); };
  void vy( double value ) { itsV.py(value); };
  void vz( double value ) { itsV.pz(value); };

  // temperature
  double T() const { return itsT; };
  void T(double value) { itsT=value; };

  // QGP fraction
  double QGPfrac() const { return itsQGPfrac; };
  void QGPfrac(double value) { itsQGPfrac = value; };
};
\end{boxedverbatim}
 
~\\
In the evolution, we access this data. In #MARTINI::evolve#, we determine the position of the parton that we want to evolve:\\

\begin{boxedverbatim}
x = plist[0]->at(i).x();                    // x value of position in [fm]
y = plist[0]->at(i).y();                    // y value of position in [fm]
z = plist[0]->at(i).z();                    // z value of position in [fm]
\end{boxedverbatim}\\

~\\
Now determine the temperature in that cell (#t# is the current time), by reading
and interpolating the data, and passing the result to the #hydroInfo# structure,
defined in #Basics.h#. It simply stores the temperature, flow velocity, and QGP
fraction values for the cell we are in. 

The reading and interpolating is done in #HydroSetup::getHydroValues#.

~\\
\begin{boxedverbatim}
hydroInfo = hydroSetup->getHydroValues(x, y, z, t, hydroXmax, hydroZmax, hydroTau0, 
                                       hydroDx, hydroDz, hydroDtau, hydroWhichHydro, 
                                       lattice);

\end{boxedverbatim}\\

~\\
In the 2+1D case #getHydroValues# reads the 8 values at the corners of a rectangle
around the position $(x,y,\tau)$ (after converting $z$ and $t$ to $\tau$).
Then it interpolates linearly to get the value of $T$, $v_x$, $v_y$ 
at the position $(x,y,\tau)$.

In the 3+1D case, the procedure is the same, only that we have a 4 dimensional
rectangle around the point $(x,y,z,\tau)$ now, 
so that we have 16 values from which we interpolate. 

The ``front bottom left corner'' of the rectangle in space is given by
~\\

\begin{boxedverbatim}        
ix = floor((hydroXmax+x)/hydroDx+0.0001); 
iy = floor((hydroXmax+y)/hydroDx+0.0001); 
iz = floor((hydroZmax+z)/hydroDz+0.0001); 
\end{boxedverbatim}\\

~\\
Note that #x# and #y# run from #-hydroXmax# to #+hydroXmax#
and #ix# and #iy# from #0# to #2*hydroXmax#,
hence the #hydroXmax#+#x# or #y# for both. Analogously for #z#.

The position on the #tau# grid is
~\\

\begin{boxedverbatim}        
itau = floor((tau-hydroTau0)/hydroDtau+0.0001);
\end{boxedverbatim}\\

~\\
From these positions we reach the other corners by going one step in the different
directions.

The flow velocity needs additional transformations, depending on $\eta$.
In the 2+1D case we need to do the following:
\begin{align}
  \beta_x &= v_x/\cosh(\eta) = v_x\,\tau/t\\
  \beta_y &= v_y/\cosh(\eta) = v_y\,\tau/t\\
  \beta_z &= z/t\,,
\end{align}
where $\eta=0.5\,\ln\left(\frac{t+z}{t-z}\right)$.

In the Nonaka 3+1D case we need to transform back from the 
$\mathbf{\tilde{v}}=(\tilde{v}_x,\tilde{v}_y,\tilde{v}_\eta)$
that is in the data files 
(this is just the inverse of Eqs.(4) in \cite{Nonaka:2006yn}):
\begin{align}
  \beta_x &= \tilde{v}_x\,\cosh(\tilde{y})/\cosh(\tilde{y}+\eta)\\
  \beta_y &= \tilde{v}_y\,\cosh(\tilde{y})/\cosh(\tilde{y}+\eta)\\
  \beta_z &= (\tilde{v}_\eta+\tanh(\tilde{y}))/(1+\tilde{v}_\eta\,\tanh(\eta))
\end{align}
where $\tilde{y}=0.5\,\ln\left(\frac{1+\tilde{v}_\eta}{1-\tilde{v}_\eta}\right)$.

In the Schenke 3+1D case this is not necessary.

All this is done in #getHydroValues# and the flow velocity that is needed
to boost to the cell's rest frame is passed to #MARTINI#.


%%% Local Variables: 
%%% mode: latex
%%% TeX-master: "manual"
%%% End: 
