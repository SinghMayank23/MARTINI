\chapter{Fragmentation}
\label{fragmentation}\index{Fragmentation}

The alternative fragmentation mechanism does not need MARTINI to keep track of the color structure of the shower.
Instead, we connect color strings in the end, when a parton exits the QGP phase or its evolution stops.
It is more likely that a high momentum parton connects to a close-by medium parton than a jet parton which went in the other direction 
initially. Assuming that the elastic collisions in the plasma randomize the color structure anyway, this should be a more realistic description.

So once a parton is done with its evolution we go into its cell and determine all its neighbours, be it jets or medium 
partons. Of course, since the medium is described by hydrodynamics there are no individual medium partons in the code yet.
So we add them to the cell. How many do we add? That is determined solely by the temperature. 
The density of gluons and quarks per degree of freedom are given by
\begin{align}
  n_g^1(T) &= \frac{4\pi}{(2\pi)^3}\int_0^\infty p^2 \frac{1}{\exp(p/T)-1} dp = \frac{\zeta(3)}{\pi^2}T^3\,,\\
  n_{u,d,s,\bar{u},\bar{d},\bar{s}}^1(T) &= \frac{4\pi}{(2\pi)^3}\int_0^\infty p^2 \frac{1}{\exp(p/T)+1} dp = \frac{3 \zeta(3)}{4 \pi^2}T^3\,.
\end{align}
Now, quarks and anti-quarks have $2\,N_c$ degrees of freedom per flavor, gluons have $2\,(N_c^2-1)$, so that the densities are
\begin{align}
  n_g(T) &= \frac{16\,\zeta(3)}{\pi^2}T^3\,,\\
  n_{u,d,s,\bar{u},\bar{d},\bar{s}}(T) &= \frac{9\,\zeta(3)}{2 \pi^2}T^3\,,
\end{align}
for $N_c=3$.
The number of medium partons that we add per cell is then given by
\begin{align}
  N_{g}(T) &= V\, n_{g}(T)\,,\\
  N_{u,d,s,\bar{u},\bar{d},\bar{s}}(T) &= V\, n_{u,d,s,\bar{u},\bar{d},\bar{s}}(T)\,,
\end{align}
where $V$ is the volume of that cell. We should take the size of a hydro cell with constant temperature.
We could also allow partons from neighboring cells to connect to the jet parton - maybe that is a good test.

We can now sample the number from some probability distribution with average $N$
or just add the exact number per cell. I'd say for now it is sufficient not to sample.
Then we sample the position within the cell (from a uniform distribution) and the momentum from the corresponding thermal 
distribution. Note that the momentum of the medium parton has then to be boosted into the 'lab-frame'.

Once the medium partons are added we can start connecting strings. we have to make sure that the group of partons that we
submit to the PYTHIA fragmentation routine is overall color neutral, i.e. that all strings are connected.

So we start with one of the jet partons in the cell and randomly select a partner for it (all other partons in the cell, 
medium or jet get the same probability).
 
If the jet parton is a quark, we can connect its string to either a gluon or an anti-quark. If we happen to choose an anti-quark,
we are done with this jet parton. If we choose a gluon, then we also have to connect the gluon to either another gluon or an anti-quark.
So, choosing gluons demands continuing more strings.

The same happens if the jet parton was an anti-quark. Just that the chain has to end with a quark now.

If the first jet parton is a gluon, we need to connect it to both a quark and anti-quark and possibly intermediate gluons.

When this chain is terminated, we move on to the next jet parton
that hasn't been connected and do the same. If we are left with a jet parton but no possible partners, 
we sample one (or two if it is a gluon) more medium parton(s) (randomly u,d, or s (anti-)quark), connect it/them to it and finish.   

To do all this effecitvely, we should first sort all jet partons into groups determined by their position, i.e., by the cell they are in.
Then we have $N_{\rm cells}$ sub-lists for which we can do the above procedure.



%%% Local Variables: 
%%% mode: latex
%%% TeX-master: "manual"
%%% End: 
