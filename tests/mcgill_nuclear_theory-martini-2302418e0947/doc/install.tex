\chapter{Installation}
\label{install}\index{Installation}
\section{MARTINI}
First just copy stuff in a directory. Details will follow once I know how it is distributed.
#MARTINI# is the name of the directory with subdirs #main#, #pythia#, #hydro#, possibly #lhapdf# and #doc#. 

\section{LHAPDF}
To get nuclear effects like shadowing to work MARTINI requires the Les Houches Accord Parton Distribution Function interface.
LHAPDF provides a unified and easy to use interface to modern PDF sets. It is designed to work not only with individual PDF 
sets but also with the more recent multiple "error" sets. It can be viewed as the successor to PDFLIB, incorporating many
of the older sets found in the latter, including pion and photon PDFs. 
Download LHAPDF from #http://www.hepforge.org/downloads/lhapdf# and install following the instructions.

After extracting the archive (to anywhere you like), in the main folder do\\
#./configure --prefix=<absolute path to MARTINI>/lhapdf#\\
Then\\
#make#\\
and\\
#make install#\\
#<absolute path to MARTINI># is e.g. #/homes/nazca/schenke/pbswork/MARTINI#.

This will install LHAPDF as a sub folder into the MARTINI folder.
Should you decide to install LHAPDF into a different folder you have to adjust the Makefiles.
For one, change #MARTINI/main/src#
so that you include the right path: change #-I ../../lhapdf/include# to your path to LHAPDF.
Also adjust the Makefile in #MARTINI/main# so that #-L../lhapdf/lib -lLHAPDF# will be #-L <your path>/lib -lLHAPDF#.
Finally, you need to set a global variable to locate the dynamic library. If using bash edit the file 
#.bashrc# in your home directory and add:
#export LD_LIBRARY_PATH=<absolute path to MARTINI>/lhapdf/lib#.

Really, PYTHIA is the one that uses LHAPDF directly. But as long as MARTINI knows where it is PYTHIA will do too.

You can also install LHAPDF as root under #/usr/share#, make sure however that you adjust the Makefiles accordingly.


\section{PYTHIA}
PYTHIA is located in #MARTINI/pythia#. Since we are using a slightly modified version of PYTHIA 8.1, it is at this point not
recommended (in fact, it won't work) to put another version here. We will work on allowing for easy exchange of the PYTHIA
version. To get everything running on a new mashine for the first time, before compiling MARTINI go to the #pythia# folder and do:\\
#make clean#\\
#./configure#\\
#make#\\
Now PYTHIA is ready.

\section{MARTINI}
Finally you can compile MARTINI. Just go to #MARTINI/main# and do\\
#make#\\

To enforce a new compilation do\\
#make clean#\\
#make#\\
or do #make -B#
 


%%% Local Variables: 
%%% mode: latex
%%% TeX-master: "manual"
%%% End: 
