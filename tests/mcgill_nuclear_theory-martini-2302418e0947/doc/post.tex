\chapter{Post processing}
\label{post}\index{Post processing}
\section{When using the original main.cpp}
At this point in time (svn revision 18, 2010/02/12), the output files of one set of runs are:\\
~\\
{\small
\begin{boxedverbatim}
distances.dat           distribution of travelled distances [fm] of quarks and gluons with errors   
distancesGluonPhi.dat   as above for gluons, colums are phi bins=0-15 15-30 30-45 45-60 60-75 75-90
distancesQuarkPhi.dat   same for quarks
gluonphi.dat            distribution of gluons vs. p_T [GeV] in different phi bins with errors
quarkphi.dat            same for quarks
partons.dat             distributions of quarks, gluons vs. p_T (no error)
pi0.dat                 p_T distribution of pi0's with error
pi0phi.dat              p_T distribution of pi0's in phi bins with error  
pi+.dat                 p_T distribution of pi+'s with error
pi-.dat                 p_T distribution of pi-'s with error
hplus.dat               p_T distribution of positive hadrons with error 
hminus.dat              p_T distribution of negative hadrons with error
protons.dat             p_T distribution of protons with error
anti-protons.dat        p_T distribution of anti-protons with error
photons.dat             p_T distribution of all photons with error
K0.dat                  p_T distribution of K_0 with error
K0S.dat                 p_T distribution of K_0 short with error 
K0L.dat                 p_T distribution of K_0 long with error
K+.dat                  p_T distribution of K_+ with error
K-.dat                  p_T distribution of K_- with error 
electrons.dat           p_T distribution of e- with error
positrons.dat           p_T distribution of e+ with error
mu.dat                  p_T distribution of mu- with error
muplus.dat              p_T distribution of mu+ with error
rho0.dat                p_T distribution of rho_0 with error
omega.dat               p_T distribution of omega with error
eta.dat                 p_T distribution of eta meson with error
phi.dat                 p_T distribution of phi meson with error
JPsi.dat                p_T distribution of J/Psi with error
theta.dat               binning of quarks in theta=atan(pt/pl) bins 
x.dat                   initial positions of all partons in x [fm]
y.dat                   initial positions of all partons in y [fm]
xy.dat                  same in y,x (in that order) and the initial momentum [GeV] in x-direction
r.dat                   initial distance from x=y=0 [fm]
test.dat                shows random seed and progress during the run (not very important)
\end{boxedverbatim}
}
~\\

There are several c-shell scripts to extract important information from these files.
Note that what I called error above is actually just the sum of squares of entries from individual runs.
It is used to compute the standard deviation in post-processing.

Now about the scripts:
\begin{itemize}
  \item #makeplot.csh#: run #./makeplot.csh N#, where #N# is the number of ``sub-runs'' to combine into the final result.
    So if you had started 10 actual runs with say 10000 events each, you will find sub-folders #1# to #10# have been created.
    Now running #./makeplot.csh 10# will combine these 10, compute the standard deviation and write the output ($p_T$ spectrum of $\pi^0$s with standard 
    deviation) into the file #pi0.dat#. This will be the average over all 100000 events.
    Using a number smaller than ten will only include the folders up to that number.
    Will also generate #pijoined# using the result for pp collisions stored in pi0-pp.dat that you have to copy into the current folder before.
    This is just the #pi0.dat# that you obtained the same way in a pp run and renamed and copied it here.
    #pijoined# has $p_T$ in GeV in the first column, then the pp result and the error in the second and third respectively, then 
    $p_T$ in GeV again (column 4 = column 1 - make sure that that is true, otherwise the binning in the pp and the AA calculation was different), and
    then the columns AA result as in #pi0.dat#.

    Needs #template_plot.csh# and #add.awk# and will create #iplot.csh# and execute it.

  \item #makeDistanceplot.csh#: same principle. no errors. output: #distances.dat#. columns: distance in [fm], quark distribution, gluon distribution

    Needs #template_DistancePlot.csh# and #addDistances.awk#, creates #iDistancePlot.csh# and runs it.

  \item #makeDistancephiplot.csh#: same principle. no errors. output: #distancesQuarkPhi.dat#. columns: distance in [fm], quark distribution in the 
    different phi bins. 

    Needs #template_Distancephiplot.csh# and #addDistancePhi.awk#, creates #iDistancephiplot.csh# and runs it.

  \item #makeDistancephiplotGluon.csh#: same principle. no errors. output: #distancesGluonPhi.dat#. columns: distance in [fm], gluon distribution in the 
    different phi bins. 

    Needs #template_DistancephiplotGluon.csh# and #addDistancePhi.awk#,\\ creates #iDistancephiplotGluon.csh# and runs it.

  \item #makeKaonPlot.csh#: like #makeplot.csh# but for $K^+$. creates #K+.dat#, and #kaonjoined# if #K+-pp.dat# is there 
    (same as with #makeplot.csh#).
    
    Needs #template_kaon_plot.csh# and #add.awk# and will create #ikaonplot.csh# and execute it.

  \item #makeK-Plot.csh#: like #makeplot.csh# but for $K^-$. creates #K-.dat#, and #K-joined# if #K--pp.dat# is there 
    (same as with #makeplot.csh#).
    
    Needs #template_K-_plot.csh# and #add.awk# and will create #iK-plot.csh# and execute it.
  
  \item #makephiplot.csh#: same principle. Will combine the results for $\pi^0$ in phi bins. creates #pi0phi.dat#.
    Will also generate #piphijoined# using the result for pp collisions stored in pi0-pp.dat that you have to copy into the current folder before.
    #piphijoined# has $p_T$ in GeV in the first column, then the pp result and the error in the second and third respectively, then 
    $p_T$ in GeV again (column 4 = column 1 - make sure that that is true, otherwise the binning in the pp and the AA calculation was different), and
    then the columns with the different phi-bins as in #pi0phi.dat#.

    Needs #template_phiplot.csh# and #addphi.awk# and will create #iphiplot.csh# and execute it. Includes columns with standard deviation.
   
  \item #makephotonplot.csh#: like #makeplot.csh# but for photons. creates #photons.dat#.
    Will also generate #photonsjoined# using the result for pp collisions stored in photons-pp.dat that you have to copy into the current folder before.
    #photonsjoined# has $p_T$ in GeV in the first column, then the pp result and the error in the second and third respectively, then 
    $p_T$ in GeV again (column 4 = column 1 - make sure that that is true, otherwise the binning in the pp and the AA calculation was different), and
    then the columns AA result as in #photons.dat#.
    
    Needs #template_photon_plot.csh# and #add.awk# and will create #iphotonplot.csh# and execute it.

  \item #makePi+Plot.csh#: like #makeplot.csh# but for $\pi^+$. creates #pi+.dat#, and #pi+joined# if #pi+-pp.dat# is there 
    (same as with #makeplot.csh#).
    
    Needs #template_pi+_plot.csh# and #add.awk# and will create #ipi+plot.csh# and execute it.
\end{itemize}

These are all routines so far. Now let me give an example on how to plot the results:
This will plot the $R_{AA}$ for $\pi^0$ for two different impact parameters using #gnuplot#\\
~\\
{\small
\begin{boxedverbatim}
set term post eps enhanced color font "Helvetica,26"
set output 'raa.eps'

set key top left Left
set key reverse
set key box
set key width -13
set key spacing 1.2
set xlabel 'p_T [GeV]'
set xrange [0:14]
set yrange [0.0000000001:1]
set style fill transparent solid 0.3 noborder

set size 1,1.3
set multiplot
set size 1,0.7
set ytics (0,0.1,0.2,0.3,0.4,0.5,0.6,0.7,0.8,0.9)
plot\
 '../data/phenix-prl-pi0-raa-central.txt' u 1:2:5 w e t 
 '{/Helvetica=22 PHENIX Au+Au 200 GeV, 0-10 % central}' pt 7 ps 2 lw 3 lt -1 lc -1,\
'pijoinednonaka17r039ptmin4.dat' every ::7::34 u ($1):($5/$2):(0.25):(($6/$2+$3*$5/($2*$2))) 
notitle w boxxyerrorbars lw 3 lt -1 lc rgbcolor "#44AA44",\
 'pijoinednonaka17r039ptmin4.dat'  every ::7::35 u ($1-0.25):($5/$2) 
 t '{/Helvetica=22 AMY + 3+1D Hydro {/Symbol a}_s=0.39 b=2.4 fm}' 
 w steps lw 6 lt 2 lc rgbcolor "#117711",\
'pijoinednonaka17el03ptmin2.dat'  every ::7::11 u ($1):($5/$2):(0.25):(($6/$2+$3*$5/($2*$2))) 
notitle w boxxyerrorbars lw 3 lt -1 lc rgbcolor "#BB4444",\
 'pijoinednonaka17el03ptmin2.dat'  every ::7::12 u ($1-0.25):($5/$2) 
 t '{/Helvetica=22 AMY + elastic + 3+1D Hydro {/Symbol a}_s=0.3 b=2.4 fm}' 
 w steps lw 3 lt -1 lc rgbcolor "#BB1111",\
'pijoinednonaka17el03ptmin4.dat'  every ::12::34 u ($1):($5/$2):(0.25):(($6/$2+$3*$5/($2*$2))) 
notitle w boxxyerrorbars lw 3 lt -1 lc rgbcolor "#BB4444",\
 'pijoinednonaka17el03ptmin4.dat'  every ::12::35 u ($1-0.25):($5/$2) 
 t '' w steps lw 3 lt -1 lc rgbcolor "#BB1111",\
 'pijoined03ptmin4nucb2.4-av'  every ::12::35 u ($1-0.25):($5/$2) 
 t '' w steps lw 3 lt -1 lc rgbcolor "#BB11BB",\
 '../data/phenix-prl-pi0-raa-central.txt' u 1:2:5 w e notitle pt 7 ps 2 lw 3 lt -1 lc -1

unset xlabel
set label '{/Symbol p}^0 R_{AA}' at graph -0.13, graph -0.15 rotate left
unset xtics
set size 1,0.6
set origin 0,0.648
set ytics (0,0.1,0.2,0.3,0.4,0.5,0.6,0.7,0.8,0.9,1)
plot\
 '../data/phenix-prl-pi0-raa-20-30.txt' u 1:2:5 w e 
 t '{/Helvetica=22 PHENIX Au+Au 200 GeV, 20-30% central}' pt 7 ps 2 lw 3 lt -1 lc -1,\
'pijoinednonaka17r039b7.5ptmin4.dat'  every ::7::34 u ($1):($5/$2):(0.25):(($6/$2+$3*$5/($2*$2))) 
notitle w boxxyerrorbars lw 3 lt -1 lc rgbcolor "#44AA44",\
'pijoinednonaka17r039b7.5ptmin4.dat'  every ::7::35 u ($1-0.25):($5/$2) 
t '{/Helvetica=22 AMY + 3+1D Hydro {/Symbol a}_s=0.39 b=7.5 fm}' 
w steps lw 6 lt 2 lc rgbcolor "#117711",\
'pijoinednonaka17el03b7.5ptmin4.dat'  every ::7::34 u ($1):($5/$2):(0.25):(($6/$2+$3*$5/($2*$2))) 
notitle w boxxyerrorbars lw 3 lt -1 lc rgbcolor "#BB4444",\
 'pijoinednonaka17el03b7.5ptmin4.dat'  every ::7::35 u ($1-0.25):($5/$2) 
 t '{/Helvetica=22 AMY + elastic + 3+1D Hydro {/Symbol a}_s=0.3 b=7.5 fm}' 
w steps lw 3 lt -1 lc rgbcolor "#BB1111",\
 '../data/phenix-prl-pi0-raa-20-30.txt' u 1:2:5 w e notitle pt 7 ps 2 lw 3 lt -1 lc -1
unset multiplot
pause -1

\end{boxedverbatim}

~\\
Note that all lines should be in one up to the ``#,\#''. I only write it this way to fit on a page.
The files beginning with #pijoined# are just the #pijoined# that we got as an output and then renamed to indicate which run they are from.
File beginning with #phenix# are experimental data.




%%% Local Variables: 
%%% mode: latex
%%% TeX-master: "manual"
%%% End: 
